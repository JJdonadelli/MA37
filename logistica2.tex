\documentclass{article}
\usepackage[utf8]{inputenc}
\usepackage{amsmath}

\begin{document}

\section*{Crescimento da População de Ovelhas na Tasmânia}

Os seguintes dados foram obtidos para o crescimento de uma população de ovelhas introduzida em um novo ambiente na ilha da Tasmânia (adaptado de J. Davidson, ``On the Growth of the Sheep Population in Tasmania,'' Trans. R. Soc. S. Australia 62 (1938): 342--346).

\[
\begin{array}{c|ccccccc}
t \, (\text{ano}) & 1814 & 1824 & 1834 & 1844 & 1854 & 1864 \\
\hline
P(t) & 125 & 275 & 830 & 1200 & 1750 & 1650 \\
\end{array}
\]

\bigskip

\textbf{a) Faça uma estimativa de $M$ através da plotagem de $P(t)$.}

Para estimar $M$, o tamanho máximo da população (ou capacidade de suporte), podemos plotar os dados $P(t)$ em função de $t$ e observar o valor para o qual a população parece se estabilizar.

\bigskip

\textbf{b) Plote $\ln \left[ \frac{P}{M - P} \right]$ contra $t$. Se uma curva logística parecer razoável, estime $rM$ e $t_*$.}

Se assumirmos que o crescimento segue uma curva logística, então a solução do modelo logístico é:

\[
P(t) = \frac{M}{1 + B e^{-r M (t - t_*)}}
\]

ou de forma equivalente:

\[
\ln \left( \frac{P}{M - P} \right) = r M (t - t_*) 
\]

Portanto, ao plotar:

\[
y = \ln \left( \frac{P}{M - P} \right)
\]

em função de $t$, devemos obter aproximadamente uma linha reta, cuja inclinação é $rM$ e cujo intercepto nos fornece $t_*$. 

Os passos são:
\begin{itemize}
    \item Estimar $M$ a partir do gráfico da população;
    \item Calcular $y = \ln \left( \frac{P}{M - P} \right)$ para cada dado de $P$;
    \item Plotar $y$ contra $t$ e ajustar uma reta;
    \item A inclinação da reta será $rM$;
    \item O ponto em que $y = 0$ corresponde a $t = t_*$.
\end{itemize}

\end{document}
