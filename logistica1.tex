\documentclass{article}
\usepackage[utf8]{inputenc}
\usepackage{amsmath}

\begin{document}

\section*{Modelo de Difusão Social}

Sociólogos reconhecem um fenômeno chamado \textit{difusão social}, que é a propagação de uma informação, uma inovação tecnológica ou uma moda cultural entre uma população. 

Os membros da população podem ser divididos em duas classes: aqueles que possuem a informação e aqueles que ainda não a possuem. 

Em uma população fixa de tamanho conhecido $N$, é razoável supor que a taxa de difusão seja proporcional ao número de indivíduos que possuem a informação vezes o número de indivíduos que ainda não a receberam.

Se $X(t)$ denota o número de indivíduos que possuem a informação em uma população de tamanho $N$, então o modelo matemático para a difusão social é dado por:

\[
\frac{dX}{dt} = k X (N - X)
\]

onde $t$ representa o tempo e $k$ é uma constante positiva.

\bigskip

\textbf{a) Resolva o modelo e mostre que ele leva a uma curva logística.}

Separando variáveis:

\[
\frac{dX}{X (N - X)} = k dt
\]

Decompondo em frações parciais:

\[
\frac{1}{X (N - X)} = \frac{1}{N} \left( \frac{1}{X} + \frac{1}{N - X} \right)
\]

Integrando:

\[
\frac{1}{N} \left( \ln |X| - \ln |N - X| \right) = k t + C
\]

\[
\ln \frac{X}{N - X} = N k t + C'
\]

onde $C' = N C$.

\[
\frac{X}{N - X} = A e^{N k t}
\]

onde $A = e^{C'}$.

\[
X = \frac{N A e^{N k t}}{1 + A e^{N k t}}
\]

ou equivalentemente:

\[
X = \frac{N}{1 + B e^{-N k t}}
\]

onde $B = \frac{N}{X_0} - 1$ e $X_0 = X(0)$.

\bigskip

\textbf{b) Em que instante a informação se espalha mais rapidamente?}

A taxa de difusão é:

\[
\frac{dX}{dt} = k X (N - X)
\]

A taxa máxima ocorre quando:

\[
X = \frac{N}{2}
\]

Substituindo na solução:

\[
\frac{N}{2} = \frac{N}{1 + B e^{-N k t}}
\]

\[
1 + B e^{-N k t} = 2
\]

\[
B e^{-N k t} = 1
\]

\[
e^{-N k t} = \frac{1}{B}
\]

\[
t = \frac{1}{N k} \ln B
\]

\bigskip

\textbf{c) Quantas pessoas eventualmente receberão a informação?}

Quando $t \to \infty$:

\[
e^{-N k t} \to 0
\]

\[
X \to N
\]

Portanto, \textbf{eventualmente toda a população receberá a informação}.

\end{document}
