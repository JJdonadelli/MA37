\documentclass[a4paper,12pt]{article}
\usepackage[utf8]{inputenc}
\usepackage{amsmath, amssymb}
\usepackage{graphicx}
\usepackage{enumitem}

\title{Capítulo 2 - Modelagem em Teoria dos Grafos}
\author{}
\date{}

\begin{document}

\maketitle

\section{Exemplo Introdutório}

Serviços como o Google Maps oferecem a possibilidade de encontrar a ``rota mais curta'' entre dois pontos geográficos \( A \) e \( B \). Essa noção de ``mais curta'' pode se referir à menor distância, menor tempo ou outras métricas (evitar pedágios, rodovias, etc). A modelagem desse problema naturalmente nos leva à teoria dos grafos.

\section{Teoria dos Grafos}

Um \textbf{grafo} é uma coleção de vértices (ou nós) conectados por arestas (ou arcos). As definições a seguir formalizam diferentes tipos de grafos.

\subsection*{Definição 2.1: Grafo Simples Não-Direcionado e Grafo Simples Direcionado}

Seja \( V \) um conjunto. O par \( G = (V, E) \) é um \textbf{grafo simples não-direcionado} se e somente se:
\[
E \subseteq \mathcal{P}(V) \quad \text{e} \quad \forall a \in E, \ |a| = 2.
\]

O par \( G = (V, E) \) é um \textbf{grafo simples direcionado} se e somente se:
\[
E \subseteq V^2 \quad \text{e} \quad \forall a = (a_1, a_2) \in E, \ a_1 \neq a_2.
\]

\subsection*{Definição 2.3: Vizinhança}

O vértice \( v \) é vizinho de \( u \) se \( uv \in E(G) \). A vizinhança de \( u \) é:
\[
N_G(u) := \{ v \in V(G) \mid uv \in E(G) \}.
\]

\subsection*{Definição 2.5: Caminhada, Trilha, Caminho}

Uma sequência finita alternada de vértices e arestas é uma:

\begin{itemize}[noitemsep]
  \item \textbf{Caminhada} de comprimento \( n \): pode repetir vértices e arestas.
  \item \textbf{Trilha}: uma caminhada sem repetição de arestas.
  \item \textbf{Caminho}: uma trilha sem repetição de vértices.
\end{itemize}

\subsection*{Definição 2.7: Grafo Ponderado}

Um grafo ponderado é um triplo \( G = (V, E, c) \) tal que \( (V, E) \) é um grafo e \( c : E \rightarrow \mathbb{R} \) é a função de custo associada às arestas.

\subsection*{Definição 2.8: Distância}

A distância entre dois vértices \( a \) e \( b \) em \( G \) é:
\[
\mathrm{dist}_G(a, b) := \min \left\{ \sum_{e \in E(w)} c(e) \mid w \text{ é uma caminhada de } a \text{ até } b \right\}.
\]

\section{Modelagem do Problema da Conexão Mais Curta}

Seja o grafo \( G = (V, E, c) \), onde:

\begin{itemize}[noitemsep]
  \item \( V \) representa interseções (cruzamentos) de ruas.
  \item \( E \) representa os trechos de rua (com direção, se necessário).
  \item \( c : E \rightarrow \mathbb{R}_0^+ \) representa o custo de percorrer cada trecho.
\end{itemize}

O problema de menor caminho entre dois pontos \( A, B \in V \) é então:

\subsection*{Problema 2.9: Menor Caminho}

\textbf{Dado:} Grafo \( G = (V, E, c) \) e dois vértices \( A, B \in V \).\\
\textbf{Determinar:} Um caminho \( p \) de \( A \) até \( B \) tal que \( c(p) = \mathrm{dist}_G(A, B) \).

\section{Algoritmo de Dijkstra}

\textbf{Objetivo:} Calcular \( \mathrm{dist}_G(A, v) \) para todos \( v \in V \setminus \{A\} \) e reconstruir o menor caminho até \( v \).

\subsection*{Pseudocódigo do Algoritmo}

\begin{enumerate}[noitemsep]
  \item Inicialize: \( C \leftarrow \emptyset \), \( O \leftarrow V \), \( l(A) \leftarrow 0 \), \( l(v) \leftarrow \infty \) para \( v \neq A \).
  \item Enquanto \( O \neq \emptyset \):
    \begin{enumerate}[noitemsep]
      \item Escolha \( v \in O \) com menor \( l(v) \).
      \item Mova \( v \) para \( C \), atualize \( \mathrm{dist}_G(A, v) = l(v) \).
      \item Para cada vizinho \( w \in N_G(v) \):
      \[
      \text{se } l(w) > l(v) + c(v, w) \text{ então } l(w) \leftarrow l(v) + c(v, w), \quad \text{pre}_G(w) \leftarrow v.
      \]
    \end{enumerate}
\end{enumerate}

Ao final, \( \mathrm{dist}_G(A, v) \) contém a menor distância de \( A \) a \( v \), e \( \text{pre}_G(v) \) permite reconstruir o caminho.

\end{document}
