\documentclass{article}
\usepackage[utf8]{inputenc}
\usepackage[brazil]{babel}
\usepackage{amsmath,amssymb, amsfonts, amsthm}

\usepackage{graphicx}
\usepackage{geometry}
\usepackage{tikz}
\usetikzlibrary{arrows.meta, positioning}
\usepackage{indentfirst}
\geometry{margin=2.5cm}
\usepackage{pgfplots}

\usepackage{booktabs}
\usepackage{bm}


\usepackage[colorlinks=true,
            linkcolor=blue,
            citecolor=blue,
            filecolor=blue,
            urlcolor=blue,
            pdfborder={0 0 0}]{hyperref}



\usepackage{pgfplots}
\pgfplotsset{compat=1.17}
\usepackage{caption}
\usepackage{subcaption}




            
\title{\textsc{Equações de Diferença Lineares de Ordem 1 e 2}} \author{Notas de aula} \date{}

\begin{document}

\maketitle

\tableofcontents

\section{Equações de diferença lineares de primeira ordem}

Considere uma equação de diferença linear de primeira ordem da forma:
\begin{equation}
x_{n+1} = a x_n + b,
\end{equation}
onde $a$ e $b$ são constantes reais, e $x_n$ representa a sequência de interesse.

A solução geral é
\[
   x(n) =
   \begin{cases}
     x_0 + bn & \text{se } a=1\\
     x_0a^n+  b\frac{1-a^n}{1-a} & \text{se } a\neq1 .
   \end{cases}
\]
Caso $a=a(n)$ e $b=b(n)$ a 
solução geral é
\[
  x(n) =
  x_0 + \sum_{i=1}^n \frac{b_i}{a_is_i}
\]
onde $s_i$ é a solução da parte homogênea $x_{n+1} = a x_n$.
\subsection{Ponto de equilíbrio}

Um ponto de equilíbrio, ou ponto fixo, $x^*$ é um valor constante tal
que, se $x_n = x^*$ para algum $n$, então $x_{n+1} = x^*$. Ou seja, a
sequência permanece constante ao longo do tempo.  No caso homogêneo
linear, frequentemente $x^* = 0$.

Para determinar $x^*$, resolvemos:
\begin{equation}
x^* = a x^* + b.
\end{equation}
Reorganizando:
\begin{equation}
x^*(1 - a) = b \quad \Rightarrow \quad x^* = \frac{b}{1 - a}, \quad \text{desde que } a \neq 1.
\end{equation}


\subsubsection{Convergência}

A convergência de uma solução $x(n)$ de uma equação de diferença de
ordem 1 refere-se ao comportamento da sequência à medida que
$n \to \infty$. Diz-se que a solução converge para um ponto $x^*$ se
\[
\lim_{n \to \infty} x(n) = x^*.
\]
Se a sequência gerada pela iteração converge, o limite é um ponto fixo
da função.


\subsubsection{Estabilidade}
Dizemos que um ponto de equilíbrio é \textit{estável} se soluções que
começam próximas a ele permanecem próximas  todo tempo. Se, além
disso, elas convergem para o equilíbrio, ele é \textit{assintoticamente
  estável}.
\begin{itemize}
\item \textbf{Estabilidade:} Para todo $\varepsilon > 0$, existe
  $\delta > 0$ tal que se $|x(0) - x^*| < \delta$, então
    \(
    |x(n) - x^*| < \varepsilon, \quad \forall n \geq 0.
    \)
    \item \textbf{Estabilidade assintótica:} Além da estabilidade, tem-se
    \(
    \lim_{n \to \infty} x(n) = x^*.
    \)
\end{itemize}
Seja $x^*$ ponto de equilíbrio. A estabilidade de $x^*$ está
relacionada ao comportamento da sequência $x_n$ quando perturbada
ligeiramente. Definimos a perturbação como $e_n = x_n -
x^*$. Substituindo na equação original:

\begin{align*}
x_{n+1} &= a x_n + b = a(x^* + e_n) + b = a x^* + a e_n + b \\
        &= x^* + a e_n \quad \text{(pois $x^*$ satisfaz $x^* = a x^* + b$)} \\
\Rightarrow e_{n+1} &= x_{n+1} - x^* = a e_n.
\end{align*}

Portanto, a perturbação evolui segundo:
\begin{equation}
e_n = a^n e_0.
\end{equation}

Uma equação de recorrência é estável se, para qualquer
$\varepsilon >0$, existe um $\delta >0$ tal que
$|x_0 -y_0|<\delta \Rightarrow |n(n) - y(n)| < \varepsilon$, i.e.,
pequenas perturbações na condição inicial não causam crescimento
assintoticamente indefinido na solução (perturbações iniciais pequenas
geram diferenças pequenas em todas as iterações).

\paragraph{Critério de estabilidade:}
\begin{itemize}
\item Se $|a| < 1$, então $e_n \to 0$ quando $n \to \infty$: o ponto de
  equilíbrio é \textbf{assintoticamente estável}.
\item Se $|a| = 1$, então $e_n$ não converge a zero: o ponto de
  equilíbrio é \textbf{neutro ou instável}, dependendo do valor de
  $e_0$, a estabilidade é marginal.
\item Se $|a| > 1$, então $e_n \to \infty$ ($x(n)$ diverge); o ponto de
  equilíbrio é \textbf{instável}.
\end{itemize}


\section{Equações de diferença lineares de segunda ordem}

Consideramos equações de diferença lineares, homogêneas, com coeficientes constantes, de ordem 2:

\begin{equation}\label{eq:2aordemhomo}
x_{n+2} + a x_{n+1} + b x_n = 0, \quad \text{com } a, b \in \mathbb{R}
\end{equation}

Nesse caso, não é difícil verificar que valem:
\begin{enumerate}
\item \emph{Se $A,r\in\mathbb R$ então $x(n) = A r^n$ é solução.}
\item \emph{Se $f(n),\,g(n)$ são soluções então $f(n)+g(n)$ é solução.}
\end{enumerate}


\subsection{Equação característica}

Buscamos soluções da forma $x_n = r^n$. Isso leva à equação característica:
\begin{equation}\label{eq:eqcarac} 
r^2 + a r + b = 0
\end{equation}

As raízes dessa equação determinam a forma geral da solução.

\paragraph{Polinômio característico}
Queremos reescrever a equação \eqref{eq:2aordemhomo} como um sistema de
equações de primeira ordem. Definimos as variáveis auxiliares:
\[
y_n = x_n, \quad z_n = x_{n+1}.
\]
Com isso, obtemos:
\[
y_{n+1} = z_n,
\]
e, substituindo na equação original:
\[
z_{n+1} = -a z_n - b y_n.
\]

Assim, o sistema equivalente de primeira ordem é:
\begin{equation}
\begin{cases}
y_{n+1} = z_n, \\
z_{n+1} = -a z_n - b y_n.
\end{cases}
\end{equation}

\paragraph{Forma Matricial}

Podemos expressar esse sistema em forma matricial como:
\begin{equation}
\begin{bmatrix}
y_{n+1} \\
z_{n+1}
\end{bmatrix}
=
\begin{bmatrix}
0 & 1 \\
-b & -a
\end{bmatrix}
\begin{bmatrix}
y_n \\
z_n
\end{bmatrix}.
\end{equation}

Denotando
\[
\bm{u}_n = \begin{bmatrix} y_n \\ z_n \end{bmatrix},
\quad \text{e} \quad
A = \begin{bmatrix} 0 & 1 \\ -b & -a \end{bmatrix},
\]
o sistema é compactamente escrito como:
\[
\bm{u}_{n+1} = A \bm{u}_n.
\]
Iterando obtemos \[\bm{u}_{n} = A^n \bm{u}_0.\]

A matriz $A$ é chamada de \textbf{matriz Jacobiana} do sistema. O
polinômio característico da matriz $A$ é dado por:
\begin{equation}
p(\lambda) = \det(A - \lambda I) = 
\det\left(
\begin{bmatrix}
-\lambda & 1 \\
-b & -a-\lambda
\end{bmatrix}
\right) =\lambda^2 +a\lambda +b.
\end{equation}
ou seja, o polinômio da equação característica \eqref{eq:eqcarac}.

\paragraph{Estabilidade} A estabilidade do sistema é determinada pelos
autovalores da matriz $A$, que são as raízes do polinômio
característico:
\begin{itemize}
\item Se todos os autovalores satisfazem $|\lambda| < 1$, o ponto de
  equilíbrio (a origem) é \textbf{assintoticamente estável}.
\item Se algum autovalor satisfaz $|\lambda| > 1$, o sistema é
  \textbf{instável}.
\item Se todos os autovalores satisfazem $|\lambda| \leq 1$ e os
  autovalores com $|\lambda| = 1$ são simples (não repetidos), o
  sistema é \textbf{estável} (mas não necessariamente assintoticamente
  estável).
\end{itemize}

Dependendo do sinal de \(\Delta= a^2 - 4b\), as formas canônicas de
Jordan (ou diagonalização) de \(A\), que são úteis para calcular a
potência $A^n$, são:

\begin{itemize}
\item \(\displaystyle \Delta>0\): dois autovalores reais distintos
  \(
  \lambda_{1,2} = \tfrac12 ({-a\pm\sqrt{\Delta}}).
  \)
  Nesse caso \(A\) é diagonalizável sobre \(\mathbb{R}\):
  \[
    A \sim
    \begin{bmatrix}
      \lambda_1 & 0 \\[3pt]
      0 & \lambda_2
    \end{bmatrix}
    \text{ de modo que } A^n =  P^{-1} \begin{bmatrix}
      \lambda_1^n & 0 \\[3pt]
      0 & \lambda_2^n 
    \end{bmatrix}P
  \]
  onde $P$ é a matriz cujas colunas são os autovetores associados a
  $\lambda_1$ e $\lambda_2$ e $\sim$ é para semelhança de matriz.
  
\item \(\displaystyle \Delta=0\): um autovalor real (com
  multiplicidade 2) \( \lambda = -\tfrac12 {a}.\)
  
  \emph{Se \(A\) for diagonalizável}, ou seja, o sistema linear
  \( (A-\lambda I)\mathbf{x} = \mathbf{0} \) tem conjunto solução de
  dimensão 2, então \(A=\lambda I\) e, imediatamente,
  \(A^n = (\lambda I)^n = \lambda^n\,I\).
  Senão, \emph{se \(A\) não for diagonalizável} temos (forma de Jordan)
  \[
    A \sim J = 
    \begin{bmatrix}
      \lambda & 1\\
      0       & \lambda
    \end{bmatrix},
  \]
  onde \(J=\lambda I+N\) e \(N^k=0\) para todo $k\geq 2$. Então
  \[
    J^n = (\lambda I + N)^n
    = \sum_{k=0}^{n} \binom{n}{k}\lambda^{\,n-k}N^k
    = \lambda^n I + n\,\lambda^{\,n-1}N.
  \]
  Como \(N=J-\lambda I\), concluímos
  \[
    J^n
    = \lambda^n I + n\,\lambda^{\,n-1}(J-\lambda I).
  \]
  Voltando à matriz \(A\), usando \(A=PJP^{-1}\), obtemos, usando que $J^n = \lambda^n
  \begin{bmatrix}
    1&n/\lambda\\0&1
  \end{bmatrix}$
  \[
    \begin{aligned}
      A^n  & = P\,J^n\,P^{-1} = P\, (\lambda^n I + n\,\lambda^{\,n-1}(J-\lambda I)) \,P^{-1} \\
      &=
      \lambda^n (PIP^{-1})+n\lambda^{n-1} (PNP^{-1}) \\ &
      = \lambda^n I + n\,\lambda^{\,n-1}(A-\lambda I).
    \end{aligned}\]
  Em outras palavras,
  \[
    \boxed{
      A^n = \lambda^n I \;+\; n\,\lambda^{\,n-1}\bigl(A-\lambda I\bigr),
      % \quad
      % \lambda = -\frac{a}{2},\quad \Delta=0.
    }
  \]
  
  
\item \(\displaystyle \Delta<0\): autovalores complexos conjugados
  \(
  \lambda = \alpha \pm i\beta,
  \quad
  \alpha = -\tfrac{a}{2},\;
  \beta = \tfrac{\sqrt{-\Delta}}{2}.
  \) 
  Não é possível diagonalização em $\mathbb R$, mas há forma de Jordan (bloco de rotação–dilatação);  \(A\) é
  semelhante a um bloco de rotação–dilatação \( A = P J P^{-1} \):
  \[
    A \sim J = 
    \begin{bmatrix}
      \alpha  & -\beta \\[3pt]
      \beta   &  \alpha
    \end{bmatrix}.
  \]
  
  Define-se \( r = \sqrt{\alpha^2+\beta^2} = \sqrt{b}\) e
  \( \theta = \arctan\!\Bigl(\frac{\beta}{\alpha}\Bigr) \) com
  \(\theta\in ( \tfrac {-\pi}2 , \tfrac{\pi}2).  \) 
  
  
  \(\theta\) é tal que
  \( \alpha = r \cos\theta \) e \( \beta = r \sin\theta \) assim,
  podemos reescrever 
  \[
    J = r
    \begin{bmatrix}
      \cos\theta & -\sin\theta \\
      \sin\theta & \cos\theta
    \end{bmatrix} = r R(\theta)
  \]
  onde \(R(\theta)\) é a matriz de rotação de ângulo \(\theta\).
  
  Queremos calcular \(A^n = P^{-1} J^n P\).  Usando as propriedades
  de potências de matrizes, temos:
  \[
    J^n = (r R(\theta))^n = r^n (R(\theta))^n = r^n  R(n\theta)
  \]
  pois \(R(\theta)\) é uma matriz de rotação, disso temos
  \( (R(\theta))^n = R(n\theta) \), pois a composição de \(n\)
  rotações de \(\theta\) é uma rotação de \(n\theta\).
  
  Finalmente,
  \[
    A^n = (P J P^{-1})^n = P J^n P^{-1} =  P
    \begin{bmatrix}
      r^n\cos(n\theta) & -r^n\sin(n\theta) \\
      r^n\sin(n\theta) & r^n\cos(n\theta)
    \end{bmatrix}
    P^{-1}.
  \]
  
\end{itemize}
    
\subsection{ Tipos de soluções e estabilidade}

De modo mais pragmático, podemos resumir caso a caso a descrição
anterior do seguinte modo de acordo coma as raízes da equação
característica
\begin{equation*}
r^2 + a r + b = 0.
\end{equation*}

\subsubsection{Raízes reais e distintas: $r_1 \neq r_2 \in \mathbb{R}$}
A solução geral é 
\begin{equation*}
  x(n) = A_1 r_1^n + A_2 r_2^n
\end{equation*}
com $A_1$ e $A_2$ determinados pelas condições iniciais $x_0$ e $x_1$
\begin{equation}
  \begin{cases}
    A_1 +A_2 &= x_0 \\
    A_1r_1 +A_2r_2 &= x_1 
  \end{cases}\label{eq:iniciais}
\end{equation}

\begin{itemize}
\item Se $|r_1| < 1$ e $|r_2| < 1$: solução  converge para o ponto de equilíbrio, o sistema é  \textbf{assintoticamente estável}.
\item Se algum $|r_i| > 1$: solução diverge,  cresce exponencialmente,  o sistema é \textbf{instável}.
\item Se algum $|r_i| = 1$, o sistema é \textbf{liminarmente estável}, e a solução pode ser limitada mas não converge.
\end{itemize}

\subsubsection{Raízes reais e iguais: $r_1 = r_2 = r$}
\begin{equation*}
  x(n) = (A_1 + A_2 n) r^n
\end{equation*}
com $A_1$ e $A_2$ determinados pelas condições inicias $x_0$ e $x_1$
\begin{equation}
  \begin{cases}
    A_1  &= x_0 \\
    A_1  + A_2 r &= x_1 
  \end{cases}
\end{equation}

\begin{itemize}
\item Se $|r| < 1$: decaimento para ponto de equilíbrio, temos \textbf{estabilidade assintótica}.
\item Se $|r| = 1$: crescimento linear, a solução diverge $\Rightarrow$ \textbf{instável}.
\item Se $|r| = 1$, o termo $n \lambda^n$ cresce em módulo: \textbf{instável}.
\end{itemize}


\subsubsection{Raízes complexas conjugadas: $\alpha \pm\mathrm{ i }\beta $}
Se o discriminante do polinômio característico \( r^2 + a r + b = 0 \)
é negativo, as raízes são complexos conjugados
\[
  \alpha \pm \mathrm{i} \beta
  =
  r  \mathrm{e}^{\pm \mathrm{i}\theta}
  =
  r(\cos \theta \pm \mathrm{i} \sin\theta)
\]
com $r= \sqrt{\alpha^2+\beta^2} =\sqrt{b}$, $\alpha=r\cos \theta$,
$\beta=r\sin\theta$ e $\theta \in(-\pi/2,\pi/2)$.  A solução ainda é
\( x(n) = A_1 r_1^n + A_2 r_2^n \) e pode ser escrita como
\[
  \begin{aligned}
    x(n) & = A_1 (\alpha + \mathrm{i} \beta)^n +  A_2 (\alpha - \mathrm{i} \beta)^n  \\
     & = r^n \left(  A_1 (\cos \theta + \mathrm{i} \sin\theta)^n +  A_2 (\cos \theta - \mathrm{i} \sin\theta)^n\right)  \\
     & = b^{n/2} \left(  A_1 (\cos (n\theta) + \mathrm{i} \sin(n\theta)) +  A_2 (\cos(n \theta) - \mathrm{i} \sin(n\theta))\right)  \\
     & = b^{n/2} \left(  (A_1+A_2)\cos (n\theta) + (A_1-A_2) \mathrm{i} \sin(n\theta) \right)  \\
     & = b^{n/2} \left(  (A_1 + A_2)\cos (n\theta) + (A_1-A_2) \mathrm{i} \sin(n\theta) \right) 
  \end{aligned}
\]
com $A_1$ e $A_2$ determinados pelas condições inicias
$x_0, x_1 \in\mathbb R$, portanto $x_n\in\mathbb R$ para todo $n$. Para
que a solução $x_n$ seja real para todo $n$, impomos que $A_2$ é o
conjugado de $A_1$.
\begin{equation*}
  \begin{cases}
    A_1 +A_2 &= x_0 \\
    A_1r_1 +A_2r_2 &= x_1 
  \end{cases}
\end{equation*}
de modo que do caso $n=0$ temos $A_1+A_2 \in\mathbb R$. Além disso,
$A_1r +A_2 \bar r \in \mathbb R$ implica em $A_2 = \bar A_1$ (deixamos
uma demonstração no final deste manuscrito. Assim
\[
  x(n) =  b^{n/2} \left(  A \cos (n\theta) + B \sin(n\theta) \right) 
\]
com $A,B\in\mathbb R$ constantes.


A solução é um termo oscilatório (cosseno e seno) multiplicado por $r^n$.
\begin{equation*}
x(n) = r^n \left( A \cos(n\theta) + B \sin(n\theta) \right)
\end{equation*}

\begin{itemize}
  \item Se $r < 1$: oscilações decrescentes (\textbf{assintoticamente estável}).
  \item Se $r = 1$: oscilações tem amplitude constante: o sistema é
    \textbf{estável} mas não assintoticamente estável (\textbf{estável
      marginalmente}).
  \item Se $r > 1$: oscilações crescentes (\textbf{instável}).
\end{itemize}

\subsubsection{Critério de estabilidade}

O ponto de equilíbrio $x^* $ satisfaz $x^*+ a x^*+ b x^*=0$ logo se
$1+a+b\neq 0$ então $x^*=0$ é ponto de equilíbrio \textbf{estável
  assintoticamente} se as raízes da equação característica satisfazem:
\begin{equation*}
|r_1| < 1 \quad \text{e} \quad |r_2| < 1 \text{ ou } |r| < 1.
\end{equation*}


A região de estabilidade no plano $(a,b)$ é caracterizada pelas seguintes condições simultâneas:
\begin{equation}
\begin{cases}
|b| < 1, \\
|a| < 1 + b, \quad \text{se} \quad b \geq 0, \\
|a| < 1 - b, \quad \text{se} \quad b \leq 0.
\end{cases}
\end{equation}

Geometricamente, a região é um losango (ou diamante) limitado pelas retas:
\[
a = 1+b, \quad a = -1-b, \quad a = 1-b, \quad a = -1+b,
\]
com $-1 < b < 1$.


\begin{center}
\begin{tikzpicture}[scale=2]

% Eixos
\draw[->] (-2,0) -- (2,0) node[right] {$a$};
\draw[->] (0,-1.5) -- (0,1.5) node[above] {$b$};

% Região de estabilidade (losango)
\fill[blue!20] 
(0,1) -- (1,0) -- (0,-1) -- (-1,0) -- cycle;

% Bordas do losango
\draw[thick, blue] (0,1) -- (1,0) -- (0,-1) -- (-1,0) -- cycle;

% Linhas de referência
\draw[dashed] (-1,1) -- (1,-1); % a = 1 - b
\draw[dashed] (-1,-1) -- (1,1); % a = 1 + b

% Anotações
\node at (1.2,0) {$a$};
\node at (0,1.3) {$b$};

\node at (0.2,1) [above right] {$b=1$};
\node at (0.2,-1) [below right] {$b=-1$};

\node at (1,0.2) [right] {$a=1$};
\node at (-1,0.2) [left] {$a=-1$};

\node[blue] at (0,0.5) {Região de Estabilidade};

\end{tikzpicture}
\end{center}


\section{Equações não homogênea de coeficientes constantes}

Considere uma equação de diferença linear de segunda ordem na forma
escalar:
\begin{equation*}
x_{n+2} + a_1 x_{n+1} + a_0 x_n = b,
\end{equation*}
onde $a_0$, $a_1$ e $b$ são constantes reais. Esse tipo de equação pode
ser reescrito como um sistema de primeira ordem, o que facilita a
análise. A forma matricial equivalente é  
\begin{equation*}
  \bm{x}_{n+1} = A \bm{x}_n + \bm{b},
  \quad \text{com} \quad
  \bm{x}_n = \begin{bmatrix} x_n \\ x_{n+1} \end{bmatrix}, 
  \quad 
  A = \begin{bmatrix}
    0 & 1 \\
    - a_0 & - a_1
  \end{bmatrix}, \quad
  \bm{b} = \begin{bmatrix}
    0 \\
    b
  \end{bmatrix}.
\end{equation*}

Um ponto de equilíbrio $\bm{x}^*$ satisfaz:
\[
\bm{x}^* = A \bm{x}^* + \bm{b} \quad \Rightarrow \quad (I - A)\bm{x}^* = \bm{b}.
\]
Se a matriz $(I - A)$ é invertível, então o ponto de equilíbrio é:
\[
\bm{x}^* = (I - A)^{-1} \bm{b}.
\]
No caso homogêneo ($b = 0$), como vimos, temos $\bm{x}^* = \bm{0}$.

A estabilidade é determinada pelos autovalores da matriz $A$, que são
as raízes do polinômio característico da equação:
\[
\lambda^2 + a_1 \lambda + a_0 = 0.
\]

Denotando as raízes por $\lambda_1$ e $\lambda_2$, temos os seguintes casos:

\begin{itemize}
    \item \textbf{Estável:} se $|\lambda_1| < 1$ e $|\lambda_2| < 1$, todas as soluções convergem para o ponto de equilíbrio.
    
    \item \textbf{Instável:} se pelo menos um dos autovalores tem módulo maior que 1.
    
    \item \textbf{Liminarmente estável:} se $|\lambda_i| \leq 1$ para
      ambos, mas algum tem módulo exatamente 1. Neste caso, a
      estabilidade depende da natureza do autovalor (simples ou
      múltiplo) e da solução.
    
    \item \textbf{Oscilações:} ocorrem quando os autovalores são
      complexos, ou seja, a equação tem solução com termos oscilatórios
      (como seno e cosseno).
\end{itemize}


\section{Sistemas de equações de ordem 1}

Vimos que a equação de diferença linear homogênea com coeficientes
constantes $x_{n+2} + a x_{n+1} + b x_n = 0$,
\( a, b \in \mathbb{R} \) é equivalente a 
ao  sistema
\begin{equation}
\begin{cases}
y_{n+1} = z_n, \\
z_{n+1} = -a z_n - b y_n.
\end{cases}
\end{equation}

Por outro lado, dado sistema de equações lineares  de primeira ordem
\begin{equation}
\label{eq:sistema}
\begin{cases}
y_{n+1} = a_{11} y_n + a_{12} z_n, \\
z_{n+1} = a_{21} y_n + a_{22} z_n,
\end{cases}
\end{equation}
tomamos \( z_n = \frac{1}{a_{12}} (y_{n+1} - a_{11} y_n)\), assumindo
$a_{12} \neq 0$, avançamos um passo na 1ª equação
\[
y_{n+2} = a_{11} y_{n+1} + a_{12} z_{n+1}
\]
e substituímos \( z_{n+1} \) usando a segunda equação de
\eqref{eq:sistema}, $z_{n+1} = a_{21} y_n + a_{22} z_n$ onde o $z_n$ é
substituído pela expressão já obtida acima
\[
z_{n+1} = a_{21} y_n + a_{22} \cdot \frac{1}{a_{12}} (y_{n+1} - a_{11} y_n).
\]
Com isso temos:
\begin{align*}
y_{n+2} &= a_{11} y_{n+1} + a_{12} z_{n+1} \\
&= a_{11} y_{n+1} + a_{12} \left[ a_{21} y_n + \frac{a_{22}}{a_{12}} (y_{n+1} - a_{11} y_n) \right] \\
&= a_{11} y_{n+1} + a_{12} a_{21} y_n + a_{22} (y_{n+1} - a_{11} y_n) \\
&= (a_{11} + a_{22}) y_{n+1} + (a_{12} a_{21} - a_{11} a_{22}) y_n.
\end{align*}

Assim, obtemos a equação de segunda ordem:
\begin{equation}
\label{eq:ordem2}
y_{n+2} - (a_{11} + a_{22}) y_{n+1} - (a_{12} a_{21} - a_{11} a_{22}) y_n = 0.
\end{equation}

Novamente, o polinômio característico da matriz do sistema
\[
p(r) = \det(r I - A) = r^2 - (a_{11} + a_{22}) r + (a_{11} a_{22} - a_{12} a_{21}),
\]
coincide com o polinômio característico da equação de segunda ordem \eqref{eq:ordem2}.


O  sistema linear na forma matricial é escrito como 
\[
  \mathbf{u}_{n+1} = A \mathbf{u}_n, \quad \text{onde } \mathbf{u}_n
  = \begin{bmatrix} y_n \\ z_n \end{bmatrix},\quad A = \begin{bmatrix}
    a_{11} & a_{12} \\ a_{21} & a_{22} \end{bmatrix}.
\]

Um vetor \( \mathbf{u}^* \in \mathbb{R}^2 \) é \textbf{ponto de
  equilíbrio} do sistema se
\( \mathbf{u}_{n+1} = \mathbf{u}_n = \mathbf{u}^* \).  Substituindo na
equação do sistema:
\[
\mathbf{u}^* = A \mathbf{u}^* \quad \Rightarrow \quad (I - A) \mathbf{u}^* = \mathbf{0}.
\]
No caso homogêneo (sem termo constante), o único ponto de equilíbrio é:
\[
\mathbf{u}^* = \mathbf{0}, \quad \text{se } \det(I - A) =  (1 - a_{11})(1 - a_{22}) - a_{12} a_{21} \neq 0.
\]

Para sistemas lineares do tipo \( \mathbf{u}_{n+1} = A \mathbf{u}_n \),
a estabilidade da origem é determinada pelos autovalores
\( \lambda_1, \lambda_2 \) da matriz \( A \):

\begin{itemize}
\item A origem é \textbf{assintoticamente estável} se
  \( |\lambda_1| < 1 \) e \( |\lambda_2| < 1 \).
  \item A origem é \textbf{instável} se ao menos um autovalor tem módulo \( > 1 \).
  \item A origem é \textbf{estável (não assintótica)} se todos os
    autovalores têm módulo \( \leq 1 \) e os de módulo 1 são simples
    (não-defeituosos).
\end{itemize}

Assim, a análise de convergência (ou seja, \( \mathbf{u}_n \to 0 \)) se
reduz ao estudo espectral da matriz \( A \).



\subsection{Exemplo: Distribuição de carros entre São Paulo e Rio de Janeiro}



Uma locadora de veículos possui filiais em São Paulo e no Rio de
Janeiro, e atende agências de turismo cujos clientes frequentemente
retiram o carro em uma cidade e o devolvem na outra. Os itinerários
podem começar em qualquer uma das cidades, e a empresa deseja
determinar quanto cobrar pela conveniência da devolução em local
diferente da retirada.

Para isso, é necessário saber se o fluxo natural de devoluções garante
um número suficiente de veículos em cada cidade para atender à demanda,
ou se será necessário transportar carros entre cidades — o que implica
custos adicionais.

Segundo registros históricos, 60\% dos carros alugados em São Paulo são
devolvidos na mesma cidade, e 40\% no Rio. Já dos carros alugados no
Rio de Janeiro, 70\% retornam ao Rio, e 30\% vão para São Paulo.

\bigskip

Essas informações são representadas graficamente na
Figura~\ref{fig:transicoes}.


\begin{figure}[h!]
\centering
\begin{tikzpicture}[node distance=3cm, >=Stealth]
  \node[circle, draw, minimum size=1.5cm] (SP) {SP};
  \node[circle, draw, right of=SP, minimum size=1.5cm] (RJ) {RJ};

  % Loops
  \draw[->] (SP) edge[loop left] node[left] {60\%} (SP);
  \draw[->] (RJ) edge[loop right] node[right] {70\%} (RJ);

  % Transitions
  \draw[->] (SP) to[bend left=20] node[above] {40\%} (RJ);
  \draw[->] (RJ) to[bend left=20] node[below] {30\%} (SP);
\end{tikzpicture}
\caption{Transições de devolução de veículos entre São Paulo e Rio de
  Janeiro}
\label{fig:transicoes}
\end{figure}

Vamos desenvolver um modelo para o sistema. Seja \( n \) o número de
dias úteis. Definimos:
\begin{align*}
  S_n &= \text{número de carros em São Paulo no final do dia } n, \\
  R_n &= \text{número de carros no Rio de Janeiro no final do dia } n.
\end{align*}

Assim, os registros históricos revelam o seguinte sistema:
\begin{align*}
S_{n+1} &= 0{,}6 S_n + 0{,}3 R_n, \\
R_{n+1} &= 0{,}4 S_n + 0{,}7 R_n.
\end{align*}

\subsubsection{Valores de Equilíbrio}

Os valores de equilíbrio para o sistema são aqueles valores de
\( S_n \) e \( R_n \) para os quais não ocorre mudança no
sistema. Vamos chamar os valores de equilíbrio, se existirem, de
\( S \) e \( R \), respectivamente. Então, temos
\( S = S_{n+1} = S_n \) e \( R = R_{n+1} = R_n \) simultaneamente.

Substituindo no modelo, obtemos:
\begin{align*}
S &= 0{,}6 S + 0{,}3 R, \\
R &= 0{,}4 S + 0{,}7 R.
\end{align*}

Ou, em forma matricial:
\[
\begin{bmatrix}
0{,}4 & -0{,}3 \\
-0{,}4 & 0{,}3
\end{bmatrix}
\begin{bmatrix}
S \\ R
\end{bmatrix}
=
\begin{bmatrix}
0 \\ 0
\end{bmatrix}
\]

Observe que
\(\det(I - A) = (0{,}4)(0{,}3) - (-0{,}3)(-0{,}4) = 0{,}12 - 0{,}12 =
0\), o que implica que o sistema possui infinitos pontos de equilíbrio
diferentes de zero.


Este sistema é satisfeito sempre que \( S = \frac{3}{4} R \), ou seja,
o conjunto dos pontos de equilíbrio é \( R ( \tfrac{3}{4} , 1 ) \) para
$R \in \mathbb{R}$. Por exemplo, se a empresa possuir 7000 carros e
começar com 3000 em São Paulo e 4000 no Rio de Janeiro, então o modelo
prevê:

\begin{align*}
  S_1 &= 0{,}6 \cdot 3000 + 0{,}3 \cdot 4000 = 3000, \\
  R_1 &= 0{,}4 \cdot 3000 + 0{,}7 \cdot 4000 = 4000.
\end{align*}

Assim, o sistema permanece em \( (S, R) = (3000, 4000) \) se iniciarmos
com esses valores.

\subsubsection{Iterações com Condições Iniciais Diferentes}

Vamos agora explorar o que acontece se começarmos com valores diferentes dos de equilíbrio. Iteramos o sistema para os seguintes quatro conjuntos de condições iniciais:

\begin{center}
\begin{tabular}{|c|c|c|}
\hline
\textbf{Caso} & \textbf{São Paulo} & \textbf{Rio de Janeiro} \\
\hline
1 & 7000 & 0 \\
2 & 5000 & 2000 \\
3 & 2000 & 5000 \\
4 & 0    & 7000 \\
\hline
\end{tabular}
\end{center}

Uma solução numérica, ou tabela de valores, para cada conjunto de
valores iniciais é mostrada na Figura correspondente (Figura 1.23 no
original).

\subsubsection{Sensibilidade às Condições Iniciais e Comportamento em Longo Prazo}

Em cada um dos quatro casos, dentro de uma semana o sistema está muito
próximo do valor de equilíbrio \( (3000,\ 4000) \), mesmo na ausência
de qualquer carro em um dos dois locais. Nossos resultados sugerem que
o valor de equilíbrio é estável e insensível aos valores iniciais.

Com base nessas explorações, somos levados a prever que o sistema se
aproxima do equilíbrio em que \( \frac{3}{7} \) da frota acaba em São
Paulo e os \( \frac{4}{7} \) restantes no Rio de Janeiro. Essa
informação é útil para a empresa: conhecendo os padrões de demanda em
cada cidade, a empresa pode estimar quantos carros precisa transportar.

Na lista de exercícios, pedimos que você explore o sistema para
determinar se ele é sensível aos coeficientes nas equações para
\( S_{n+1} \) e \( R_{n+1} \).


%\paragraph{Caso 1:} \textbf{São Paulo = 7000, Rio de Janeiro = 0}

\begin{table}[h]
%\centering
\begin{minipage}[h]{.45\linewidth}
\caption{Evolução da Frota - Caso 1}
\begin{tabular}{cccc}
\toprule
$n$ & $S_n$ & $R_n$ \\
\midrule
0 & 7000 & 0 \\
1 & 4200 & 2800 \\
2 & 3360 & 3640 \\
3 & 3108 & 3892 \\
4 & 3032.4 & 3967.6 \\
5 & 3009.72 & 3990.28 \\
6 & 3002.916 & 3997.084 \\
7 & 3000.875 & 3999.125 \\
\bottomrule
\end{tabular}
%
\end{minipage}
\begin{minipage}[h]{.45\linewidth}
\begin{tikzpicture}[scale=.9]
\begin{axis}[
    xlabel={$n$ (dias)},
    ylabel={Número de carros},
    legend style={at={(0.5,-0.2)},anchor=north},
    title={Evolução da Frota - Caso 1},
    xtick={0,...,7},
    grid=major
]
\addplot coordinates {
    (0,7000)(1,4200)(2,3360)(3,3108)(4,3032.4)(5,3009.72)(6,3002.916)(7,3000.875)
};
\addplot coordinates {
    (0,0)(1,2800)(2,3640)(3,3892)(4,3967.6)(5,3990.28)(6,3997.084)(7,3999.125)
};
\legend{São Paulo, Rio de Janeiro}
\end{axis}
\end{tikzpicture}
\end{minipage}
\end{table}

%\paragraph{Caso 2:}  \textbf{São Paulo = 5000, Rio de Janeiro = 2000}

\begin{table}[h!]
  \begin{minipage}[h]{.45\linewidth}
\caption{Evolução da Frota - Caso 2}
\begin{tabular}{cccc}
\toprule
$n$ & $S_n$ & $R_n$ \\
\midrule
0 & 5000 & 2000 \\
1 & 3600 & 3400 \\
2 & 3180 & 3820 \\
3 & 3054 & 3946 \\
4 & 3016.2 & 3983.8 \\
5 & 3004.86 & 3995.14 \\
6 & 3001.458 & 3998.542 \\
7 & 3000.437 & 3999.563 \\
\bottomrule
\end{tabular}
\end{minipage}
\begin{minipage}[h]{.45\linewidth}
\begin{tikzpicture}
\begin{axis}[
    xlabel={$n$ (dias)},
    ylabel={Número de carros},
    legend style={at={(0.5,-0.2)},anchor=north},
    title={Evolução da Frota - Caso 2},
    xtick={0,...,7},
    grid=major
]
\addplot coordinates {
    (0,5000)(1,3600)(2,3180)(3,3054)(4,3016.2)(5,3004.86)(6,3001.458)(7,3000.437)
};
\addplot coordinates {
    (0,2000)(1,3400)(2,3820)(3,3946)(4,3983.8)(5,3995.14)(6,3998.542)(7,3999.563)
};
\legend{São Paulo, Rio de Janeiro}
\end{axis}
\end{tikzpicture}
\end{minipage}
\end{table}

%\paragraph{Caso 3:}  \textbf{São Paulo = 2000, Rio de Janeiro = 5000}

\begin{table}[h!]
  \begin{minipage}[h]{.45\linewidth}
\caption{Evolução da Frota - Caso 3}
\begin{tabular}{cccc}
\toprule
$n$ & $S_n$ & $R_n$ \\
\midrule
0 & 2000 & 5000 \\
1 & 2700 & 4300 \\
2 &  2910 &4090 \\
3 & 2973 & 4027 \\
4 & 2991.9 & 4008.1 \\
5 & 2997.57 &   4002.43\\
6 & 2999.271 & 4000.729 \\
7 &  2999.781  &4000.219\\
\bottomrule
\end{tabular}
\end{minipage}
\begin{minipage}[h]{.45\linewidth}
\begin{tikzpicture}
\begin{axis}[
    xlabel={$n$ (dias)},
    ylabel={Número de carros},
    legend style={at={(0.5,-0.2)},anchor=north},
    title={Evolução da Frota - Caso 3},
    xtick={0,...,7},
    grid=major
]
\addplot coordinates {
    (0,2000)(1,2700)(2,2910)(3,2973)(4,2991.9)(5,2997.57)(6,2999.271)(7,2999.781)
};
\addplot coordinates {
    (0,5000)(1,4300)(2,4090)(3,4027)(4,4008.1)(5,4002.43)(6,4000.729)(7,4000.219)
};
\legend{São Paulo, Rio de Janeiro}
\end{axis}
\end{tikzpicture}
\end{minipage}
\end{table}


\begin{table}[h!]
  \begin{minipage}[h]{.45\linewidth}
\caption{Evolução da Frota - Caso 4}
\begin{tabular}{cccc}
\toprule
$n$ & $S_n$ & $R_n$ \\
\midrule
0 & 0 & 7000 \\
1 & 2100 & 4900 \\
2 & 2730 &4270  \\
3 & 2919 & 4081 \\
4 & 2975.7  & 4024.3 \\
5 & 2992.71 &  4007.29\\
6 & 2997.813  & 4002.187 \\
7 & 2999.344 & 4000.656 \\
\bottomrule
\end{tabular}
\end{minipage}
\begin{minipage}[h]{.45\linewidth}
\begin{tikzpicture}
\begin{axis}[
    xlabel={$n$ (dias)},
    ylabel={Número de carros},
    legend style={at={(0.5,-0.2)},anchor=north},
    title={Evolução da Frota - Caso 4},
    xtick={0,...,7},
    grid=major
]
\addplot coordinates {
  (0,0)(1,2100)(2,2730)(3,2919)(4,2975.7)(5,2992.71)(6,2997.813)(7,2999.344)
};
\addplot coordinates {
        (0,7000)(1,4900)(2,4270)(3,4081)(4,4024.3)(5,4007.29)(6,4002.187)(7,4000.656)
};
\legend{São Paulo, Rio de Janeiro}
\end{axis}
\end{tikzpicture}
\end{minipage}
\end{table}


\clearpage

\subsection{Exemplo:  Modelo Discreto de Epidemias}

Considere uma doen\c{c}a que est\'a se espalhando, como uma nova
gripe. O governo est\'a interessado em estudar e experimentar um modelo
para essa nova doen\c{c}a antes que ela se torne de fato uma
epidemia. Vamos considerar a popula\c{c}\~ao dividida em tr\^es
categorias: suscet\'iveis ($S$), infectados ($I$) e removidos ($R$).  O
modelo considerado \'{e} conhecido como SIR\footnote{os parâmetros do
  modelo SIR são de difícil obtenção. É necessário equipes
  interdisciplinares para fazer tratamento de dados.}. Fazemos as
seguintes suposi\c{c}\~oes para nosso modelo:
\begin{itemize}
\item Ningu\'em entra ou sai da comunidade e n\~ao h\'a contato com o
  exterior.
\item Cada pessoa est\'a em um dos tr\^es estados: suscet\'ivel $S$
  (pode pegar a gripe), infectado $I$ (tem a gripe e pode
  transmiti-la), ou removido $R$ (j\'a teve a gripe e n\~ao pode
  peg\'a-la novamente, incluindo os casos de \'obito).
\item Inicialmente, cada pessoa est\'a em $S$ ou $I$.
\item Uma vez que algu\'em pega a gripe neste ano, n\~ao pode peg\'a-la
  novamente.
\item A dura\c{c}\~ao m\'edia da doen\c{c}a \'e de $5/3$ semanas
  ($1\tfrac 23$ semana), durante a qual a pessoa \'{e}
  considerada infectada e pode transmitir a doen\c{c}a.
\item O per\'{\i}odo de tempo do modelo \'{e} semanal.
\end{itemize}
Além disso, a dura\c{c}\~ao da doen\c{c}a \'{e} $5/3$ semanas.

Definimos as seguintes vari\'{a}veis:
\begin{align*}
  S_n &= \text{n\'umero de suscet\'iveis ap\'os o per\'iodo } n \\
  I_n &= \text{n\'umero de infectados ap\'os o per\'iodo } n \\
  R_n &= \text{n\'umero de removidos ap\'os o per\'iodo } n
\end{align*}
e come\c{c}amos modelando $R_n$. A \emph{taxa de remoção} $\gamma$ é a
proporção de infetados removidos em um período,
$ \Delta R = \gamma I_n$, ou seja, é a proporção que indica qual fração
das pessoas infectadas sai da categoria ``infectado'' em um período de
tempo,
\[
  R_{n+1} = R_n + \gamma \cdot I_n
\]
e se $D$ é a duração média da infecção então $D\gamma =1$. No nosso
exemplo $\gamma =3/5 = 0{,}6$, ou seja, 60\% dos infectados deixam de
ser infectados a cada semana. 

A \textit{taxa de transmissão} $\beta$ mede a velocidade com que a
doença se espalha na população, ela representa a probabilidade de um
contato entre um suscetível e um infectado resultar em infecção por
unidade de tempo, assim o número esperado de novas infecções
\[
  \Delta S= - \beta S_n I_n.
\]
O termo de novos infectados por semana é $\Delta I$: $\beta S_n I_n$,
em cada semana, cada par suscetível–infectado tem uma chance de
aproximadamente $\beta$ de resultar em nova infecção diminuída da
remo\c{c}\~ao dos infectados $\gamma  I_n$
\begin{equation}
  \label{eq:taxainfeccao}
  \Delta I = (\beta S_n - \gamma)I_n
\end{equation}
se $\beta S_n > \gamma$ a doença tende a se espalhar e se se
$\beta S_n < \gamma$ tende a desaparecer. Assumimos inicialmente que
$\beta$ \'{e} constante e pode ser estimada a partir das
condi\c{c}\~oes iniciais\footnote{Por exemplo, se $I(0) = 5$ e
  $S(0) = 995$, e $I(1) = 9$, temos:
  $I(1) = I(0) - 0{,}6 \cdot I(0) + a \cdot I(0) \cdot S(0) \Rightarrow
  \beta = 0{,}001407$}

\paragraph{Observação}
$ R_0
= %\frac{\text{taxa de transmissão}}{\text{taxa de remoção}} S_0 =
\frac{\beta}{\gamma}S_0$ define o \emph{número básico de reprodução},
que mede o potencial de disseminação da doença: Se $R_0 > 1$, a doença
tende a se espalhar; se $R_0 < 1$, a epidemia tende a desaparecer. No
início $R_0 \approx (\beta/\gamma)N$. Nesse caso, \textbf{o isolamento
  social (quarentena) diminui o $\beta$ que, por sua vez, faz diminuir
  o $R_0$}.

\medskip

Nosso modelo  \'{e} ent\~ao:
\[
  \begin{cases}
    R_{n+1} = R_n + 0{,}6 I_n \\
    I_{n+1} = I_n - 0{,}6 I_n + 0{,}001407 I_n S_n \\
    S_{n+1} = S_n - 0{,}001407 S_n I_n
  \end{cases}
\]
com condi\c{c}\~oes iniciais:
\( I(0) = 5, \quad S(0) = 995, \quad R(0) = 0 \).


Esse sistema SIR pode ser resolvido iterativamente e analisado
graficamente para compreender o comportamento da epidemia.


\begin{table}[htpb]
\centering
\begin{minipage}{0.48\textwidth}\footnotesize
\centering
\begin{tabular}{|c|c|c|c|}
\hline
\textbf{Semana ($n$)} & \textbf{$S_n$} & \textbf{$I_n$} & \textbf{$R_n$} \\ \hline
0  & 995.000000 & 5.000000 & 0.000000 \\ \hline
1  & 988.000175 & 8.999825 & 3.000000 \\ \hline
2  & 975.489372 & 16.110733 & 8.399895 \\ \hline
3  & 953.377173 & 28.556492 & 18.066335 \\ \hline
4  & 915.071446 & 49.728324 & 35.200230 \\ \hline
5  & 851.045954 & 83.916821 & 65.037225 \\ \hline
6  & 750.562135 & 134.050548 & 115.387317 \\ \hline
7  & 608.999271 & 195.183083 & 195.817646 \\ \hline
8  & 441.754309 & 245.318195 & 312.927496 \\ \hline
9  & 289.277198 & 250.604389 & 460.118413 \\ \hline
10 & 187.277950 & 202.241004 & 610.481046 \\ \hline
11 & 133.987430 & 134.186921 & 731.825649 \\ \hline
12 & 108.690470 & 78.971729 & 812.337801 \\ \hline
13 & 96.613521  & 43.665640 & 859.720839 \\ \hline
14 & 90.677823  & 23.401955 & 885.920223 \\ \hline
15 & 87.692115  & 12.346490 & 899.961396 \\ \hline
16 & 86.168770  & 6.461940  & 907.369289 \\ \hline
17 & 85.385328  & 3.368218  & 911.246454 \\ \hline
18 & 84.980680  & 1.751936  & 913.267385 \\ \hline
19 & 84.771205  & 0.910249  & 914.318546 \\ \hline
20 & 84.662636  & 0.472668  & 914.864696 \\ \hline
21 & 84.606332  & 0.245372  & 915.148296 \\ \hline
22 & 84.577123  & 0.127358  & 915.295519 \\ \hline
23 & 84.561967  & 0.066099  & 915.371934 \\ \hline
24 & 84.554103  & 0.034304  & 915.411593 \\ \hline
25 & 84.550022  & 0.017803  & 915.432176 \\ \hline
\end{tabular}
\end{minipage}%
\hfill
\begin{minipage}{0.48\textwidth}\footnotesize
\centering
\begin{tabular}{|c|c|c|c|}
\hline
\textbf{Semana ($n$)} & \textbf{$S_n$} & \textbf{$I_n$} & \textbf{$R_n$} \\ \hline
26 & 84.547904 & 0.009239 & 915.442857 \\ \hline
27 & 84.546805 & 0.004795 & 915.448400 \\ \hline
28 & 84.546235 & 0.002488 & 915.451277 \\ \hline
29 & 84.545939 & 0.001291 & 915.452770 \\ \hline
30 & 84.545785 & 0.000670 & 915.453545 \\ \hline
31 & 84.545705 & 0.000348 & 915.453947 \\ \hline
32 & 84.545664 & 0.000180 & 915.454156 \\ \hline
33 & 84.545642 & 0.000094 & 915.454264 \\ \hline
34 & 84.545631 & 0.000049 & 915.454320 \\ \hline
35 & 84.545626 & 0.000025 & 915.454349 \\ \hline
36 & 84.545623 & 0.000013 & 915.454364 \\ \hline
37 & 84.545621 & 0.000007 & 915.454372 \\ \hline
38 & 84.545620 & 0.000004 & 915.454376 \\ \hline
39 & 84.545620 & 0.000002 & 915.454378 \\ \hline
40 & 84.545620 & 0.000001 & 915.454379 \\ \hline
41 & 84.545619 & 0.0000005 & 915.454380 \\ \hline
42 & 84.545619 & 0.0000003 & 915.454380 \\ \hline
43 & 84.545619 & 0.0000001 & 915.454381 \\ \hline
44 & 84.545619 & 0.00000007 & 915.454381 \\ \hline
45 & 84.545619 & 0.00000004 & 915.454381 \\ \hline
46 & 84.545619 & 0.00000002 & 915.454381 \\ \hline
47 & 84.545619 & 0.00000001 & 915.454381 \\ \hline
48 & 84.545619 & 0.00000001 & 915.454381 \\ \hline
49 & 84.545619 & 0.00000000 & 915.454381 \\ \hline
50 & 84.545619 & 0.00000000 & 915.454381 \\ \hline
\end{tabular}
\end{minipage}
\caption{Evolução das variáveis do modelo SIR ao longo das semanas (tabela dividida).}
\end{table}

\normalsize
%
%
%\begin{table}[htpb]
%\centering
%\begin{tabular}{|c|c|c|c|}
%\hline
%\textbf{Semana ($n$)} & \textbf{$S_n$} & \textbf{$I_n$} & \textbf{$R_n$} \\ \hline
%0  & 995.000000   & 5.000000    & 0.000000 \\ \hline
%1  & 988.000175   & 8.999825    & 3.000000 \\ \hline
%2  & 975.489372   & 16.110733   & 8.399895 \\ \hline
%3  & 953.377173   & 28.556492   & 18.066335 \\ \hline
%4  & 915.071446   & 49.728324   & 35.200230 \\ \hline
%5  & 851.045954   & 83.916821   & 65.037225 \\ \hline
%6  & 750.562135   & 134.050548  & 115.387317 \\ \hline
%7  & 608.999271   & 195.183083  & 195.817646 \\ \hline
%8  & 441.754309   & 245.318195  & 312.927496 \\ \hline
%9  & 289.277198   & 250.604389  & 460.118413 \\ \hline
%10 & 187.277950   & 202.241004  & 610.481046 \\ \hline
%11 & 133.987430   & 134.186921  & 731.825649 \\ \hline
%12 & 108.690470   & 78.971729   & 812.337801 \\ \hline
%13 & 96.613521    & 43.665640   & 859.720839 \\ \hline
%14 & 90.677823    & 23.401955   & 885.920223 \\ \hline
%15 & 87.692115    & 12.346490   & 899.961396 \\ \hline
%16 & 86.168770    & 6.461940    & 907.369289 \\ \hline
%17 & 85.385328    & 3.368218    & 911.246454 \\ \hline
%18 & 84.980680    & 1.751936    & 913.267385 \\ \hline
%19 & 84.771205    & 0.910249    & 914.318546 \\ \hline
%20 & 84.662636    & 0.472668    & 914.864696 \\ \hline
%21 & 84.606332    & 0.245372    & 915.148296 \\ \hline
%22 & 84.577123    & 0.127358    & 915.295519 \\ \hline
%23 & 84.561967    & 0.066099    & 915.371934 \\ \hline
%24 & 84.554103    & 0.034304    & 915.411593 \\ \hline
%25 & 84.550022    & 0.017803    & 915.432176 \\ \hline
%26 & 84.547904    & 0.009239    & 915.442857 \\ \hline
%27 & 84.546805    & 0.004795    & 915.448400 \\ \hline
%28 & 84.546235    & 0.002488    & 915.451277 \\ \hline
%29 & 84.545939    & 0.001291    & 915.452770 \\ \hline
%30 & 84.545785    & 0.000670    & 915.453545 \\ \hline
%31 & 84.545705    & 0.000348    & 915.453947 \\ \hline
%32 & 84.545664    & 0.000180    & 915.454156 \\ \hline
%33 & 84.545642    & 0.000094    & 915.454264 \\ \hline
%34 & 84.545631    & 0.000049    & 915.454320 \\ \hline
%35 & 84.545626    & 0.000025    & 915.454349 \\ \hline
%36 & 84.545623    & 0.000013    & 915.454364 \\ \hline
%37 & 84.545621    & 0.000007    & 915.454372 \\ \hline
%38 & 84.545620    & 0.000004    & 915.454376 \\ \hline
%39 & 84.545620    & 0.000002    & 915.454378 \\ \hline
%40 & 84.545620    & 0.000001    & 915.454379 \\ \hline
%41 & 84.545619    & 0.0000005   & 915.454380 \\ \hline
%42 & 84.545619    & 0.0000003   & 915.454380 \\ \hline
%43 & 84.545619    & 0.0000001   & 915.454381 \\ \hline
%44 & 84.545619    & 0.00000007  & 915.454381 \\ \hline
%45 & 84.545619    & 0.00000004  & 915.454381 \\ \hline
%46 & 84.545619    & 0.00000002  & 915.454381 \\ \hline
%47 & 84.545619    & 0.00000001  & 915.454381 \\ \hline
%48 & 84.545619    & 0.00000001  & 915.454381 \\ \hline
%49 & 84.545619    & 0.00000000  & 915.454381 \\ \hline
%50 & 84.545619    & 0.00000000  & 915.454381 \\ \hline
%\end{tabular}
%\caption{Evolução das variáveis do modelo SIR ao longo das semanas}
%\end{table}
%\clearpage
\begin{figure}[htbp]
  \centering
  \begin{tikzpicture}[scale=.75]
    \begin{axis}[
      width=13cm,
      height=7cm,
      xlabel={Semana $n$},
      ylabel={População},
      xmin=0, xmax=22,
      ymin=0, ymax=1000,
      legend pos=north east,
      grid=both,
      xtick={0,2,4,6,8,10,12,14,16,18,20,22},
      ytick={0,200,400,600,800,1000},
      title={Modelo SIR: $S_n$, $I_n$ e $R_n$}
      ]
      
      % Dados S_n
      \addplot[blue, thick] coordinates {
        (0,995)(1,988.0002)(2,975.4894)(3,953.3772)(4,915.0714)
        (5,851.046)(6,750.5621)(7,608.9993)(8,441.7543)(9,289.2772)
        (10,187.2779)(11,133.9874)(12,108.6905)(13,96.61352)(14,90.67782)
        (15,87.69211)(16,86.16877)(17,85.38533)(18,84.98068)(19,84.7712)
        (20,84.66264)(21,84.60633)(22,84.57712)
      };
      
      % Dados I_n
      \addplot[red, thick] coordinates {
        (0,5)(1,8.999825)(2,16.11073)(3,28.55649)(4,49.72832)
        (5,83.91682)(6,134.0505)(7,195.1831)(8,245.3182)(9,250.6044)
        (10,202.241)(11,134.1869)(12,78.97173)(13,43.66564)(14,23.40195)
        (15,12.34649)(16,6.46194)(17,3.368218)(18,1.751936)(19,0.910249)
        (20,0.472668)(21,0.245372)(22,0.127358)
      };
      
      % Dados R_n
      \addplot[green!60!black, thick] coordinates {
        (0,0)(1,3)(2,8.399895)(3,18.06633)(4,35.20023)
        (5,65.03722)(6,115.3873)(7,195.8176)(8,312.9275)(9,460.1184)
        (10,610.481)(11,731.8256)(12,812.3378)(13,859.7208)(14,885.9202)
        (15,899.9614)(16,907.3693)(17,911.2465)(18,913.2674)(19,914.3185)
        (20,914.8647)(21,915.1483)(22,915.2955)
      };
      
      \legend{$S_n$, $I_n$, $R_n$}
    \end{axis}
  \end{tikzpicture}
  \caption{Evolução das populações Suscetível, Infectada e Removida ao
    longo do tempo.}
\end{figure}

\begin{figure}[ht]
  \centering
  % --- S_n ---
  \begin{subfigure}{0.3\textwidth}
    \centering
    \begin{tikzpicture}[scale=.4]
      \begin{axis}[
        title={$S_n$ -- Suscetíveis},
        xlabel={Semana $n$},
        ylabel={$S_n$},
        ymin=0, ymax=1000,
        xmin=0, xmax=22,
        grid=both,
        width=12cm,
        height=6cm,
        xtick={0,2,...,22},
        ytick={0,200,...,1000},
        mark options={solid, blue},
        ]
        \addplot[
        color=blue,
        mark=*,
        thick
        ] coordinates {
          (0,995)(1,988.0002)(2,975.4894)(3,953.3772)(4,915.0714)
          (5,851.046)(6,750.5621)(7,608.9993)(8,441.7543)(9,289.2772)
          (10,187.2779)(11,133.9874)(12,108.6905)(13,96.61352)(14,90.67782)
          (15,87.69211)(16,86.16877)(17,85.38533)(18,84.98068)(19,84.7712)
          (20,84.66264)(21,84.60633)(22,84.57712)
        };
      \end{axis}
    \end{tikzpicture}
  \end{subfigure}
  %
  %\vspace{0.7cm}
  %
  % --- I_n ---
  \begin{subfigure}{0.3\textwidth}
    \centering
    \begin{tikzpicture}[scale=.4]
      \begin{axis}[
        title={$I_n$ -- Infectados},
        xlabel={Semana $n$},
        ylabel={$I_n$},
        ymin=0, ymax=300,
        xmin=0, xmax=22,
        grid=both,
        width=12cm,
        height=6cm,
        xtick={0,2,...,22},
        ytick={0,50,...,300},
        mark options={solid, red},
        ]
        \addplot[
        color=red,
        mark=*,
        thick
        ] coordinates {
          (0,5)(1,8.999825)(2,16.11073)(3,28.55649)(4,49.72832)
          (5,83.91682)(6,134.0505)(7,195.1831)(8,245.3182)(9,250.6044)
          (10,202.241)(11,134.1869)(12,78.97173)(13,43.66564)(14,23.40195)
          (15,12.34649)(16,6.46194)(17,3.368218)(18,1.751936)(19,0.910249)
          (20,0.472668)(21,0.245372)(22,0.127358)
        };
      \end{axis}
    \end{tikzpicture}
  \end{subfigure}
  %
  % \vspace{0.7cm}
  %
  % --- R_n ---
  \begin{subfigure}{0.3\textwidth}
    \centering
    \begin{tikzpicture}[scale=.4]
      \begin{axis}[
        title={$R_n$ -- Removidos},
        xlabel={Semana $n$},
        ylabel={$R_n$},
        ymin=0, ymax=1000,
        xmin=0, xmax=22,
        grid=both,
        width=12cm,
        height=6cm,
        xtick={0,2,...,22},
        ytick={0,200,...,1000},
        mark options={solid, green!60!black},
        ]
        \addplot[
        color=green!60!black,
        mark=*,
        thick
        ] coordinates {
          (0,0)(1,3)(2,8.399895)(3,18.06633)(4,35.20023)
          (5,65.03722)(6,115.3873)(7,195.8176)(8,312.9275)(9,460.1184)
          (10,610.481)(11,731.8256)(12,812.3378)(13,859.7208)(14,885.9202)
          (15,899.9614)(16,907.3693)(17,911.2465)(18,913.2674)(19,914.3185)
          (20,914.8647)(21,915.1483)(22,915.2955)
        };
      \end{axis}
    \end{tikzpicture}
  \end{subfigure}
  
  \caption{Evolução temporal de $S_n$, $I_n$ e $R_n$ com pontos
    marcados.}
\end{figure}

\subsubsection{Ponto de equilíbrio do modelo SIR discreto}

O ponto de equilíbrio é um conjunto de valores $(S^*, I^*, R^*)$ tal
que, se o sistema atinge esses valores, ele permanece neles para
sempre.

\noindent\textbf{Conservação da população} Como ninguém entra ou sai da
comunidade (população fechada), temos:
$S_n+I_n+R_n=N=\text{constante}$.

Formalmente:
\[
S_{n+1} = S_n = S^*, \quad I_{n+1} = I_n = I^*, \quad R_{n+1} = R_n = R^*
\]

As equações do sistema SIR discreto são:

\[
\begin{cases}
S_{n+1} = S_n - a S_n I_n \\
I_{n+1} = I_n - \gamma I_n + a S_n I_n \\
R_{n+1} = R_n + \gamma I_n
\end{cases}
\]

No equilíbrio, temos:
\[
\Delta S = -a S^* I^* = 0
\]
\[
\Delta I = -\gamma I^* + a S^* I^* = 0
\]
\[
\Delta R = \gamma I^* = 0
\]

Portanto, a única solução é $I^* = 0$, ou seja, não há mais infectados.
Assim, o ponto de equilíbrio é:
\[
\boxed{ (S^*, I^*, R^*) = (S^*,\ 0,\ N - S^*) }
\]
com $0 \leq S^* \leq N$.

O sistema sempre converge para uma situação em que não há mais
infectados e a população se divide entre os que nunca foram infectados
($S^*$) e os que foram infectados e se recuperaram ou morreram ($R^*$)
O valor exato de $S^*$ depende da dinâmica da epidemia, não
necessariamente todos se infectam.

\subsubsection{Estimativa de $S^*$ (número de suscetíveis no final)}

Queremos estimar $S^*$, o número final de pessoas que nunca foram
infectadas. O número básico de reprodução é
\[
R_0 = \frac{\beta }{\gamma} N.
\]
Com os valores do exemplo: \(\beta = 0{,}001407\), \(\gamma = 0{,}6\) e \(N = 1000\)
\[
R_0 = \frac{0{,}001407 \cdot 1000}{0,6} = \frac{1{,}407}{0{,}6} \approx 2{,}345
\]

Vamos relacionar diretamente \( S \) e \( R \), eliminando \( I
\). Como a população total é constante
\( \Delta S_n + \Delta I_n + \Delta R_n = 0 \) calculamos as variações:
\[
  \begin{aligned}
    \Delta S_n &= S_{n+1} - S_n = -\beta S_n I_n\\
    \Delta R_n &= R_{n+1} - R_n = \gamma I_n
\end{aligned}
\]
dividindo as duas expressões:
\[
  \frac{\Delta S_n}{\Delta R_n} = -\frac{\beta S_n I_n}{\gamma I_n} =
  -\frac{\beta}{\gamma} S_n
\]
portanto
\[
  \frac{\Delta S_n}{S_n} = 
  -\frac{\beta}{\gamma}  R_n
\]
somando ao longo do tempo (do estado inicial até o final $T$):
\[
  \sum_{n=0}^{T} \frac{\Delta S_n}{S_n} = -\frac{\beta}{\gamma} \sum_{n=0}^{T}
  \Delta R_n.
\]
O lado direito aproximamos por uma integral
$\int \tfrac 1S \mathrm{d}S = \ln(S(T)) -\ln(S(0)) =
\ln(S^*)-\ln(S_0)$, o lado direito é uma soma telescópica, o que leva
a:
\[
\ln S^* - \ln S_0 = -\frac{\beta}{\gamma}  (R^* - R_0) = -\frac{\beta}{\gamma} (N - S^*)
\]
pois \( R_0 = 0 \) e \( R^* = N - S^* \). Concluindo
\[
\boxed{  
\ln {S^*} =\ln(S_0) -\frac{\beta}{\gamma}  (N - S^*)  
}
\]
A equação acima relaciona o número final de suscetíveis \( S^* \) ao
número inicial \( S_0 \), ela é uma equação \emph{implícita} -- não
conseguimos isolar \( S^* \) de forma analítica --, por isso, para
encontrar \( S^* \), devemos recorrer a métodos \emph{numéricos} ou
\emph{gráficos}.  Contudo, uma boa aproximação prática quando $R_0 > 1$
é:
\[
  S^*/N \approx e^{-R_0 \cdot (1 - S^*/N)}
\]
Com $R_0 \approx 2,345$, é esperado que aproximadamente 15\%  da
população permaneça suscetível. No nosso exemplo, com $N = 1000$, temos:
\[
\frac{S^*}N \approx \frac{84,5}{1000} (\text{8,45\% da população})
\]
permanece suscetível.


\appendix


\section{Coeficientes conjugados}

\emph{Sejam \( A_1, A_2 \in \mathbb{C} \) tais que
\[
A_1 + A_2 = a, \quad a \in \mathbb{R},
\]
e
\[
z A_1 + \bar{z} A_2 = b, \quad b \in \mathbb{R},
\]
onde \( z \in \mathbb{C} \) e \( \bar{z} \) é o conjugado de \( z \).
Então \( A_1 \) e \( A_2 \) são conjugados um do outro.  }

\bigskip

Como \( a \in \mathbb{R} \), temos:
\[
\overline{A_1 + A_2} = A_1 + A_2.
\]
Analogamente, como \( b \in \mathbb{R} \), temos:
\[
\overline{z A_1 + \bar{z} A_2} = z A_1 + \bar{z} A_2.
\]
Expandindo o lado esquerdo:
\begin{align*}
\overline{z A_1 + \bar{z} A_2} &= \overline{z A_1} + \overline{\bar{z} A_2} \\
&= \bar{z} \, \overline{A_1} + z \, \overline{A_2}.
\end{align*}
Portanto,
\[
z A_1 + \bar{z} A_2 = \bar{z} \, \overline{A_1} + z \, \overline{A_2}.
\]
Agrupando os termos:
\begin{align*}
(z A_1 + \bar{z} A_2) - (\bar{z} \, \overline{A_1} + z \, \overline{A_2}) &= 0, \\
z (A_1 - \overline{A_2}) + \bar{z} (A_2 - \overline{A_1}) &= 0.
\end{align*}

Agora, como \( z \) e \( \bar{z} \) são, em geral, linearmente independentes sobre \(\mathbb{C}\), concluímos que:
\[
\begin{cases}
A_1 - \overline{A_2} = 0, \\
A_2 - \overline{A_1} = 0.
\end{cases}
\]
Ou seja,
\[
\begin{cases}
A_1 = \overline{A_2}, \\
A_2 = \overline{A_1}.
\end{cases}
\]

Assim, \( A_1 \) e \( A_2 \) são conjugados um do outro.

Agora, \( z = \alpha + \mathrm{i} \beta \) e
\( \bar{z} = \alpha - \mathrm{i} \beta \) não são linearmente
independentes sobre \(\mathbb{C}\) quando $\alpha = 0$ ou $\beta = 0$.
O caso $\beta =0$ implica em raiz real ($\Delta = 0$).

O caso $\alpha = 0 $ ocorre quando $x_{n+2} + bx_n =0$.  A equação
característica é
\[
r^2 + b = 0
\quad \Rightarrow \quad
r^2 = -b
\]

\textbf{Caso 1: \( b>0 \)} As raízes são \( r = \pm i\sqrt{b} \).  A
solução geral é:
\( x_n = A \cos(\sqrt{b}\, n) + B \sin(\sqrt{b}\, n) \) onde \( A \) e
\( B \) são constantes determinadas pelas condições iniciais.
    
\textbf{Caso 2: \( b<0 \)} As raízes são reais:
\( r = \pm \sqrt{-b} \).  A solução geral é:
\( x_n = A (\sqrt{-b})^n + B (-\sqrt{-b})^n \) que pode ser reescrita
como \( x_n = (\sqrt{-b})^n (A + B(-1)^n) \).

\textbf{Caso 3: \( b=0 \)} A equação se reduz a \( x_{n+2} = 0 \),
portanto, \( x_2 = x_3 = x_4 = \cdots = 0 \) e a solução é    
\[
  x_n =
  \begin{cases}
    x_0, & \text{se } n=0 \\
    x_1, & \text{se } n=1 \\
    0, & \text{se } n \geq 2
  \end{cases}
\]


\section{Exemplo: estabilidade de um sistema linear}

Considere o sistema:
\begin{equation}
\begin{bmatrix}
x_{n+1} \\
y_{n+1}
\end{bmatrix}
=
A
\begin{bmatrix}
x_n \\
y_n
\end{bmatrix}, \quad \text{com} \quad
A = 
\begin{bmatrix}
0.5 & 0.2 \\
0   & 0.8
\end{bmatrix}.
\end{equation}
Neste caso, o vetor constante $\bm{b} = \bm{0}$, então o ponto de equilíbrio é $\bm{0}$.

Para analisar a estabilidade, determinamos os autovalores de $A$:
\[
\det(A - \lambda I) = 
\begin{vmatrix}
0.5 - \lambda & 0.2 \\
0 & 0.8 - \lambda
\end{vmatrix}
= (0.5 - \lambda)(0.8 - \lambda).
\]
As raízes da equação característica são:
\[
\lambda_1 = 0.5, \quad \lambda_2 = 0.8.
\]
Como ambos os autovalores satisfazem $|\lambda| < 1$, o ponto de
equilíbrio $\bm{0}$ é \textbf{assintoticamente estável}.

\subsection*{Simulação de trajetória}

Considere a condição inicial
$\bm{x}_0 = \begin{bmatrix} 1 \\ 1 \end{bmatrix}$.  Calculamos algumas
iterações:
\[
\bm{x}_1 = A \bm{x}_0 = 
\begin{bmatrix}
0.5 & 0.2 \\
0   & 0.8
\end{bmatrix}
\begin{bmatrix}
1 \\
1
\end{bmatrix}
=
\begin{bmatrix}
0.7 \\
0.8
\end{bmatrix}, \quad
\bm{x}_2 = A \bm{x}_1 = 
\begin{bmatrix}
0.5(0.7) + 0.2(0.8) \\
0.8(0.8)
\end{bmatrix}
=
\begin{bmatrix}
0.47 \\
0.64
\end{bmatrix},
\]
e assim por diante.


\begin{center}
\begin{tikzpicture}
\begin{axis}[
    axis lines = middle,
    xlabel = {$x_n$},
    ylabel = {$y_n$},
    width=8cm,
    height=8cm,
    xmin=0, xmax=1.2,
    ymin=0, ymax=1.2,
    grid=both,
    xtick={0,0.2,...,1.2},
    ytick={0,0.2,...,1.2},
    enlargelimits=true,
    legend style={at={(1,1)}, anchor=north east}
]
\addplot[mark=*, color=blue] coordinates {
    (1.0, 1.0)
    (0.7, 0.8)
    (0.47, 0.64)
    (0.331, 0.512)
    (0.2427, 0.4096)
    (0.18449, 0.32768)
    (0.143634, 0.262144)
};
\addlegendentry{Trajetória}

\end{axis}
\end{tikzpicture}
\end{center}

A trajetória converge para a origem, confirmando a \textbf{estabilidade assintótica} do sistema.

\section{Exemplo: sistema com autovalores complexos}

Considere o sistema linear:
\begin{equation}
\bm{x}_{n+1} = A \bm{x}_n, \quad \text{com} \quad 
A = \begin{bmatrix}
0.8 & -0.6 \\
0.6 & 0.8
\end{bmatrix}.
\end{equation}

Calculamos os autovalores resolvendo $\det(A - \lambda I) = 0$:
\[
\det\begin{bmatrix}
0.8 - \lambda & -0.6 \\
0.6 & 0.8 - \lambda
\end{bmatrix}
= (0.8 - \lambda)^2 + 0.36 = 0.
\]
\[
(0.8 - \lambda)^2 = -0.36 \quad \Rightarrow \quad \lambda = 0.8 \pm 0.6i.
\]

Os autovalores são complexos conjugados com módulo:
\[
|\lambda| = \sqrt{0.8^2 + 0.6^2} = \sqrt{0.64 + 0.36} = \sqrt{1} = 1.
\]

\paragraph{Discussão da estabilidade}
Como $|\lambda| = 1$, o sistema tem soluções com comportamento
oscilatório (devido à parte imaginária) e \textbf{não converge ao ponto
  de equilíbrio}. Como não há crescimento, o sistema é \emph{estável no
  sentido de Lyapunov}, mas \textbf{não é assintoticamente estável}.

Se o módulo fosse um pouco menor que 1 (por exemplo,
$\lambda = 0.9 \pm 0.4i$), o sistema seria {assintoticamente estável
  com oscilação decrescente}.

\subsection*{Simulação da trajetória}

Seja a condição inicial:
\[
\bm{x}_0 = \begin{bmatrix} 1 \\ 0 \end{bmatrix}.
\]

Vamos iterar o sistema:
\[
\bm{x}_1 = A \bm{x}_0 = \begin{bmatrix} 0.8 \\ 0.6 \end{bmatrix}, \quad
\bm{x}_2 = A \bm{x}_1 = \begin{bmatrix} 0.8(0.8) - 0.6(0.6) \\ 0.6(0.8) + 0.8(0.6) \end{bmatrix} = \begin{bmatrix} 0.28 \\ 0.96 \end{bmatrix}, \text{etc.}
\]


\begin{center}
\begin{tikzpicture}
\begin{axis}[
    axis lines = middle,
    xlabel = {$x_n$},
    ylabel = {$y_n$},
    width=8cm,
    height=8cm,
    xmin=-1.2, xmax=1.2,
    ymin=-1.2, ymax=1.2,
    grid=both,
    xtick={-1,-0.5,...,1},
    ytick={-1,-0.5,...,1},
    legend style={at={(1,1)}, anchor=north east}
]
\addplot[mark=*, color=purple] coordinates {
    (1.0, 0.0)
    (0.8, 0.6)
    (0.28, 0.96)
    (-0.376, 0.928)
    (-0.88096, 0.336)
    (-0.9984, -0.528576)
    (-0.7008, -0.958752)
    (-0.078, -1.120)
    (0.595, -0.870)
    (0.964, -0.336)
};
\addlegendentry{Trajetória}

\end{axis}
\end{tikzpicture}
\end{center}

A trajetória gira em torno da origem com raio constante, formando uma
órbita elíptica — típico de sistemas com autovalores complexos com
$|\lambda| = 1$.

\section{Sistema com autovalores complexos e $\mathbf{b} \neq \mathbf{0}$}

Considere o sistema de equações de diferença:
\begin{equation}
\bm{x}_{n+1} = A \bm{x}_n + \bm{b}, \quad \text{com} \quad
A = \begin{bmatrix}
0.8 & -0.6 \\
0.6 & 0.8
\end{bmatrix}, \quad
\bm{b} = \begin{bmatrix}
1 \\
0
\end{bmatrix}.
\end{equation}

\paragraph{Ponto de equilíbrio}

O ponto de equilíbrio $\bm{x}^*$ satisfaz:
\[
\bm{x}^* = A \bm{x}^* + \bm{b} \quad \Rightarrow \quad (I - A)\bm{x}^* = \bm{b}.
\]

Calculamos $I - A$:
\[
I - A = \begin{bmatrix}
0.2 & 0.6 \\
-0.6 & 0.2
\end{bmatrix}, \quad
\det(I - A) = 0.2^2 + 0.6^2 = 0.04 + 0.36 = 0.4.
\]

A inversa é:
\[
(I - A)^{-1} = \frac{1}{0.4} \begin{bmatrix}
0.2 & -0.6 \\
0.6 & 0.2
\end{bmatrix}
= \begin{bmatrix}
0.5 & -1.5 \\
1.5 & 0.5
\end{bmatrix}.
\]

Logo:
\[
\bm{x}^* = (I - A)^{-1} \bm{b} =
\begin{bmatrix}
0.5 \\
1.5
\end{bmatrix}.
\]

\paragraph{Dinâmica da perturbação} Se definirmos
$\bm{e}_n = \bm{x}_n - \bm{x}^*$, então a equação para a perturbação é:
\[
\bm{e}_{n+1} = A \bm{e}_n,
\]
isto é, a mesma equação homogênea do exemplo anterior. Portanto, a
trajetória de $\bm{x}_n$ será uma espiral em torno do ponto de
equilíbrio $\bm{x}^* = \begin{bmatrix} 0.5 \\ 1.5 \end{bmatrix}$, com
raio constante (já que $|\lambda| = 1$).

\begin{center}
\begin{tikzpicture}
\begin{axis}[
    axis lines = middle,
    xlabel = {$x_n$},
    ylabel = {$y_n$},
    width=8cm,
    height=8cm,
    xmin=-1, xmax=2,
    ymin=0, ymax=3,
    grid=both,
    legend style={at={(1,1)}, anchor=north east}
]
\addplot[mark=*, color=orange] coordinates {
    (1.0, 0.0)
    (1.8, 0.6)
    (2.08, 1.68)
    (1.68, 2.64)
    (0.72, 3.04)
    (-0.064, 2.52)
    (-0.138, 1.4)
    (0.46, 0.62)
    (1.368, 0.74)
    (2.01, 1.55)
};
\addlegendentry{Trajetória}

\addplot[mark=*, color=red] coordinates {(0.5, 1.5)};
\addlegendentry{Ponto de equilíbrio}

\end{axis}
\end{tikzpicture}
\end{center}

 O sistema é \textbf{estável}, mas \textbf{não assintoticamente estável}.



\end{document}
